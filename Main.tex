\documentclass[journal=jcisd8,manuscript=article]{achemso}


%=============== Packages ===================%

\usepackage{amsmath}
\usepackage{amssymb}
\usepackage{amsfonts}
\usepackage{graphicx}
\usepackage{colortbl}
\usepackage{url}
\usepackage{tikz}
\usepackage[ruled]{algorithm2e}
\usepackage{flushend}
\usepackage{footnote}
\usepackage{threeparttable}
\usepackage[justification=centering]{caption}
\usepackage{subcaption}
\usepackage[caption=false,font=footnotesize]{subfig}
\usepackage{amsthm}
\usepackage{setspace}
\usepackage{mhchem}
\usepackage{textcomp}


%TODO: GET THIS ON WEDNESDAY MORNING!!!
\newcommand{\caconc}{$10^{-4}$}

\title{CARBONIC ANHYDRASE: KINETICS OF Zn(II) REMOVAL BY 2,6-PYRIDINECARBOXYLATE}
\author{Paul Nickerson}
\email{pvnick@ufl.edu}
\affiliation[University of Florida]
{Department of Chemistry, University of Florida, Gainesville, Fl, US}

\doublespacing

\begin{document}

%=============== Abstract ==================%
\begin{abstract}
\singlespacing{
Carbonic Anhydrase is an enzyme which catalyzes the interconversion of carbon dioxide and carbonic acid/bicarbonate. It uses a zinc cofactor noncovalently bound to the active site for its catalytic activity. In this experiment, 2,6-pyridinecarboxylate (commonly referred to as dipic) was used to strip this ion from the enzyme, and kinetics of the involved mechanisms are investigated. An equilibrium constant is established, $K_{EML}$, which describes the binding affinity of the dipic ligand for the enzyme (forming a ternary complex) at $25^\circ{C}$ and 7.4pH. $K_{EML}$ was found to have a value of $(20\pm{18}){\ }M^{-1}$. In addition, a rate constant, $k_d$, is presented which describes the velocity of conversion from the ternary complex to the inactive form of the enzyme and the covalent zinc-dipic molecule. $k_d$ was found to have a value of $(0.10\pm{0.10}){\ }mins^{-1}$. These values are compared with those found in literature, and their chemical significance is discussed. Potential sources of error for these values are explored, along with possibilities for further investigation.
}
\end{abstract}

%=============== Content ===================%
\section{Introduction}
hello world
\begin{align}
\ce{CO2 + C <=> 2CO}
\end{align}
\section{Procedure}
This experiment was performed using a Spectral Instruments model SI 440 spectrophotometer, which includes a CCD detector and fiber optic probe. This allows absorbances to be measured in situ, rather than requiring the use of a cuvette. The spectrophotometer was set to detect absorbance at 348 nm every five seconds for a total of two minutes. Device precison was set to "low," and a blank was obtained and locked. A sample of deionized water was measured to ensure absorbance stayed constant, preventing systematic errors from faulty instrumentation.

A 0.003 M aqueous pNPA solution was prepared by dissolving 25 mg solid pNPA in 1.5 mL acetone within a 250 mL Erlenmeyer flask, then slowly adding 50 mL deionized water while stirring vigorously to prevent precipitation. Assays were constructed by combining 1.7 mL deionized water, 0.3 mL HEPES buffer (0.25 M, pH 8.0), and 1.0 mL pNPA solution.

To measure the uncatalyzed assay velocity, 40 �L deionized water was added to an assay, and the absorbance was measured for two minutes. $\left( frac{dA}{dt} \right)_{uncat}$ was found to be $3.18\times10^{-5}$.

Five "runs" were performed, one for each of five dipic concentrations: 0.20 M, 0.10 M, 0.05 M, 0.032 M, and 0.016 M. For each [dipic], a 500 �L solution was prepared consisting of 250 �L aqueous bovine carbonic anhydrase ($10^{-4}$ M in 0.125 M phosphate buffer, pH 7.5) and the required volumes of phosphate buffer (0.125 M, pH 7.4) and dipic (0.4 M in 0.125 M phosphate buffer, pH 7.4) needed to dilute to the specified [dipic]. In each run, the carbonic anhydrase was added to the buffer and placed in a $25^{\circ}$C water bath to equilibrate for several minutes. Dipic was added just prior to starting measurements. After one minute, and then again at regular intervals over the course of one hour, a 40 �L aliquot was transferred from the carbonic anhydrase solution to an assay cuvette, and a $frac{dA}{dt}$ measurement was recorded.
\section{Sample Calculations}
\subsection{Catalyzed Velocity Baseline Correction}
As 
\section{Error Analysis}
To determine $k_{obs}$, readings of $ln \left(\frac{dA}{dt}\right)_{cat}$ were plotted against CA/dipic reaction time, and a least squares regression was performed. From the analysis of 0.10 M dipic, $m=-0.0936{\ }{mins}^{-1}$ and $s_m=0.0207{\ }{mins}^{-1}$. There were 5 data points taken while active CA enzyme activity was detected, so 3 degrees of freedom were used to get calculate the 95\% confidence t-factor, $t_{95}=3.182$. From this, a 95\% confidence interval was calculated:
\begin{equation*}
\begin{split}
-k_{obs}&=\left (-0.0936\pm\frac{0.0207\times{3.182}}{\sqrt{5}}\right ){\ }{mins}^{-1}\\
&=\left (-0.090\pm0.029\right ){\ }{mins}^{-1}
\end{split}
\end{equation*}

To calculate $k_{d}$ and $K_{EML}$, values of $\frac{1}{k_{obs}}$ retrieved from all five experiments were plotted against $\frac{1}{\text{[L]}_0}$. A least squares regression determined $m=0.5372{\ }mins\cdot{M}$, $s_m=0.1675{\ }mins\cdot{M}$, $b=8.513{\ }{mins}$, and $s_b=5.507{\ }{mins}$. Since 5 runs were performed, there were 3 degrees of freedom. The 95\% confidence t-factor was $t_{95}=3.182$. Therefore,
\begin{equation}\label{eqn:err_analysis_m}
\begin{split}
m_{95}&=\left (0.5372\pm\frac{0.1675\times{3.182}}{\sqrt{5}}\right ){\ }mins\cdot{M}\\
&=\left (0.50\pm0.24\right ){\ }mins\cdot{M}
\end{split}
\end{equation}
\begin{equation}\label{eqn:err_analysis_b}
\begin{split}
b_{95}&=\left (8.513\pm\frac{5.507\times{3.182}}{\sqrt{5}}\right ){\ }{mins}\\
&=\left (9\pm8\right ){\ }{mins}
 \end{split}
\end{equation}

Equation \eqref{eqn:calculating_keml} shows how to calculate $K_{EML}$ from the slope and intercept determined by Equations \eqref{eqn:err_analysis_m} and \eqref{eqn:err_analysis_b}:
\begin{equation*}
K_{EML}=\left (\frac{b}{m}\pm\lambda_{K_{EML}}\right ){\ }M^{-1}
\end{equation*}
Therefore, a 95\% confidence interval can be established as follows:
\begin{equation*}
\begin{split}
\lambda_{K_{EML}}&=\sqrt{\left (\frac{\partial{\ }K_{EML}}{\partial{\ }m}\right )^{2}\lambda_{m}^{2}+\left (\frac{\partial{\ }K_{EML}}{\partial{\ }b}\right )^{2}\lambda_{b}^{2}} \\
&=\sqrt{\left (-{\ }\frac{b}{m^{2}}\right )^{2}\lambda_{m}^{2}+\left (\frac{1}{m}\right )^{2}\lambda_{b}^{2}} \\
&=\sqrt{\left (-{\ }\frac{\left (9.0{\ }mins\right )}{\left (0.5{\ }M\cdot{mins}\right )^{2}}\right )^2\times\left (0.24{\ }M\cdot{mins}\right )^2+\left (\frac{1}{\left (0.5{\ }M\cdot{mins}\right )}\right )^2\times\left (8.0{\ }mins\right )^2} \\
&=18\ M^{-1}
\end{split}
\end{equation*}

To calculate $k_d$, use the y-intercept of the line passing through a plot of $\frac{1}{k_{obs}}$ versus $\frac{1}{\text{[L]}_0}$ as determined by Equation \eqref{eqn:err_analysis_b}. Combining that with Equation \eqref{eqn:calculating_kd},
\begin{equation*}
k_{d}=\left (\frac{1}{b}\pm\lambda_{k_{d}}\right ){\ }mins^{-1}
\end{equation*}
Therefore, a 95\% confidence interval can be established as follows:
\begin{equation*}
\begin{split}
\lambda_{k_{d}}&=\sqrt{\left (\frac{\partial{\ }k_{d}}{\partial{\ }b}\right )^{2}\lambda_{b}^{2}}\\
&=\sqrt{\left (-{\ }\frac{1}{b^{2}}\right )^{2}\lambda_{b}^{2}}\\
&=\sqrt{\left (-{\ }\frac{1}{\left (9.0{\ }\right )^{2}}\right )^2\times\left (8.0{\ }\right )^2}\\
&=0.10\ mins^{-1}
\end{split}
\end{equation*}

\section{Data and Results}

\begin{figure}[h]
  \includegraphics[scale=0.5]{./Figures/20M_dipic_readings.png}\\
  \caption{laksjdf askdfj asdkljfhaslkdjfhaklsdjfh asldkfjhasdlkjfhalksdjfhalksdjfh asldkfjhasdlkfjashdlfkajsdhflkajshdf asdfa sdiufhaisdu fas dfasd fasef asefasef asefas efasefkahsdkjfhaskldjfha sdfasldkfjhasdlfkjasdh falksdjfhalskdjfhasdlkfjah sdflkjashdlkfjah sdlkfjh aslkdjfh aslkdjfhalksdjfhalksjdh flkajsdhf alsdkjfash dflaskdjfah .}\label{fig:0.20M_dipic_readings}
\end{figure}

In the past decade, supervised activity recognition methods have been studied by many researchers, however these methods still face many challenges in real world settings. Supervised activity recognition methods assume that we are provided with labeled training examples from a set of predefined activities. Annotating and hand labeling data is a very time consuming and laborious task. Also, the assumption of consistent pre-defined activities might not hold in reality. More importantly, these algorithms do not take into account the streaming nature of data, or the possibility that the patterns might change over time. In this chapter, we will provide an overview of the state of the art \emph{unsupervised} methods for activity recognition. In particular, we will describe a scalable activity discovery and recognition method for complex large real world datasets, based on sequential data mining and stream data mining methods.

\begin{figure}[h]
  \includegraphics[scale=0.5]{./Figures/10M_dipic_readings.png}\\
  \caption{blahblah.}\label{fig:0.10M_dipic_readings}
\end{figure}

\begin{figure}[h]
  \includegraphics[scale=0.5]{./Figures/05M_dipic_readings.png}\\
  \caption{blahblah.}\label{fig:0.05M_dipic_readings}
\end{figure}

\begin{figure}[h]
  \includegraphics[scale=0.5]{./Figures/032M_dipic_readings.png}\\
  \caption{blahblah.}\label{fig:0.032M_dipic_readings}
\end{figure}

\begin{figure}[h]
  \includegraphics[scale=0.5]{./Figures/016M_dipic_readings.png}\\
  \caption{blahblah.}\label{fig:0.016M_dipic_readings}
\end{figure}

\begin{center}
    \begin{tabular}{| l | l | l | l |}
    \hline
    Day & Min Temp & Max Temp & Summary \\ \hline
    Monday & 11C & 22C & A clear day with lots of sunshine.
    However, the strong breeze will bring down the temperatures. \\ \hline
    Tuesday & 9C & 19C & Cloudy with rain, across many northern regions. Clear spells
    across most of Scotland and Northern Ireland,
    but rain reaching the far northwest. \\ \hline
    Wednesday & 10C & 21C & Rain will still linger for the morning.
    Conditions will improve by early afternoon and continue
    throughout the evening. \\
    \hline
    \end{tabular}
\end{center}
\section{Conclusion}
In this experiment, kinetics were investigated for the removal of \ce{Zn^2+} from the active site of carbonic anhydrase by 2,6-pyridinecarboxylate (dipic). Concentrations of dipic were varied for constant concentrations of carbonic anhydrase, and the resulting catalytic activities were measured periodically over the course of holoenzyme depletion using spectrophotometry.

Dipic forms an enzyme-metal-ligand complex at the carbonic anhydrase active site, and then either reverses back to the holoenzyme and dipic or proceeds to covalently bond with zinc, stripping it from the active site and converting the enzyme to the inactive form, apoCA:
\begin{align*}
\ce{CA$\cdot$Zn + L
<=>[\ce{K_{EML}}]
CA$\cdot$Zn$\cdot$L}
\end{align*}
\begin{align*}\label{eqn:kd_reaction}
\ce{CA$\cdot$Zn$\cdot$L
->[\ce{k_{d}}]
apoCA + ZnL}
\end{align*}
$K_{EML}$ was determined to be $(20\pm{18}){\ }M^{-1}$, a relatively wide error range which includes the literature value, $(7.7\pm{1.5}){\ }M^{-1}$\cite{bib:easy_peasy_values}. The enzyme-metal-ligand complex is therefore thermodynamically favorable over the separated enzyme and the dipic ion by approximately a factor of ten. $k_{d}$ was measured at $(0.10\pm{0.10}){\ }mins^{-1}$, while the accepted value from literature, $(0.43\pm{0.08}){\ }mins^{-1}$, is slightly outside of that range. What this means is that of the enzyme-metal-ligand complex extant at any point, approximately $25-40\%$ is converted into apoCA and ZnL within a minute. A consequence of these data is that the assumption that $\text{[CA$\cdot$Zn$\cdot$L]}=0$ needed to convert Equation \eqref{eqn:integrated_rate_law} into \eqref{eqn:integrated_rate_law_frac} may not necessarily be reliable. Instead, Equation \eqref{eqn:integrated_rate_law} would translate into
\begin{equation}\label{eqn:integrated_rate_law}
ln \left( \frac{\text{[CA$\cdot$Zn]}_0 - \text{[apoCA]}}{\text{[CA$\cdot$Zn]}_0} \right)
=
ln \left( \frac{\text{[CA$\cdot$Zn]}_t + \text{[CA$\cdot$Zn$\cdot$L]}}{\text{[CA$\cdot$Zn]}_0} \right)
\end{equation}
where [CA$\cdot$Zn$\cdot$L] is a constant due to steady state approximation. The final effect on the accuracy of $K_{EML}$ and $k_{d}$ is difficult to determine and should be investigated further. It may turn out to be negligible.

Besides the source of uncertainty just discussed, an obvious source of uncertainty is the subjective selection of usable data required. In order to determine values for $k_{obs}$, assay measurements were selected or discarded visually based on their conforming to a descending straight line, and due to the data set's relative sparsity it was difficult to determine the exact range of enzyme activity. One way to correct for this uncertainty is to simply collect more assay measurements. Rather than having five or six data points within the window of holoenzyme activity, using ten or twenty would increase the precision of the least squares regression. Instead of scrambling to quickly perform assays, the temperature may be lowered so that the reactions occur slower. 

One more possible source of error was the way in which the assay absorbance velocities were determined. $\frac{dA}{dt}$ was calculated by subtracting the difference in absorbance at the start and end of the two minute assay and didn't account for the uncertainty involved in individual absorbance readings. This has more to do with the limitations of the spectrophotometer software. To correct for this in the future, all absorbance readings from the two minute assay could be exported to a data file and loaded into a spreadsheet application, which can be used to establish the slope through least squares regression. The calculated error from this linear regression could then be further propagated throughout the calculations. 

%=============== Reference ===================%
\newpage
\begin{thebibliography}{9}

\bibitem{bib:ca_ph_dependence}
  Lindskog, S; Coleman, J. E.
  \emph{Proc. Nat. Acad. Sci. USA.}
  \textbf{1973},
  \emph{70},
  2505 - 2508.
  
\bibitem{bib:pdb_carbonicanhydrase}
  RCSB Protein Data Bank.
  Refined structure of human carbonic anhydrase II at 2.0 A resolution.
  http://www.rcsb.org/pdb/explore/explore.do?structureId=1CA2 (accessed Jun 16, 2014).

\bibitem{bib:lab_manual}
  Killian, B. J.
  \emph{Experiments for Physical Chemistry Laboratory},
  Summer 2014,
  Target Copy: Gainesville,
  \textbf{2014}.
  45 - 50.

\bibitem{bib:lehninger_mm}
  Nelson, D. L.; Cox, M. M.
  \emph{Lehninger Principles of Biochemistry},
  5th ed.;
  W. H. Freeman: New York,
  2008.
  195 - 200.

\bibitem{bib:quantitative_chem_anal_beer_law}
  Harris, D. C.
  \emph{Quantitative Chemical Analysis},
  7th Ed.;
  W. H. Freeman: New York,
  2006.
  381.
  
\bibitem{bib:easy_peasy_values}
  Williams K. R; Adhyaru B.
  \emph{Journal of Chemical Education}
  \textbf{2004},
  \emph{81},
  1045 - 1047.

\end{thebibliography}

\end{document} 