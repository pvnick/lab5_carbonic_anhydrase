\section{Procedure}
This experiment was performed using a Spectral Instruments model SI 440 spectrophotometer, which includes a CCD detector and fiber optic probe. The spectrophotometer was set to detect absorbance at 348 nm every five seconds for a total of two minutes. Device precison was set to "low," and a blank was obtained and locked. A sample of deionized water was measured to ensure absorbance stayed constant, preventing systematic errors from faulty instrumentation.

A 0.003 M aqueous pNPA solution was prepared by dissolving 25 mg solid pNPA in 1.5 mL acetone within a 250 mL Erlenmeyer flask, then slowly adding 50 mL deionized water while stirring vigorously to prevent precipitation. Assays were constructed by combining 1.7 mL deionized water, 0.3 mL HEPES buffer (0.25 M, pH 8.0), and 1.0 mL pNPA solution.

To measure the uncatalyzed assay velocity, 40 �L deionized water was added to an assay, and the absorbance was measured for two minutes. $\left( \frac{dA}{dt} \right)_{uncat}$ was found to be $3.18\times10^{-5}$.

Five dipic concentrations were tested: 0.20 M, 0.10 M, 0.05 M, 0.032 M, and 0.016 M. For each [dipic], a 500 �L solution was prepared consisting of 250 �L aqueous bovine carbonic anhydrase ($10^{-4}$ M in 0.125 M phosphate buffer, pH 7.5) and the required volumes of phosphate buffer (0.125 M, pH 7.4) and dipic (0.4 M in 0.125 M phosphate buffer, pH 7.4) needed to dilute to the specified [dipic]. In each run, the carbonic anhydrase and buffer were mixed and placed in a $25^{\circ}$C water bath to equilibrate for several minutes. Dipic was mixed just prior to starting measurements. After one minute, and then again at regular intervals over the course of one hour (or until $\frac{dA}{dt}$ measurements stopped decreasing), a 40 �L aliquot of CA/dipic solution was transferred to an assay cuvette, and a two-minute $\frac{dA}{dt}$ measurement was recorded. 