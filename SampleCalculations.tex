\section{Sample Calculations}
In each experiment, a 500 {\textmu}L solution ($v_{soln}=500$ {\textmu}L) was prepared consisting of 250 {\textmu}L of carbonic anhydrase solution, a certain volume ($v_{buff}$) of phosphate buffer, and, just before measurements were started, a certain volume ($v_{L}$) of 0.4 M dipic ($\text{[L]}_{stock}=0.4$ M) needed to dilute the dipic to a specific concentration ($\text{[L]}_0$). To achieve 0.10 M dipic, as per Equation \eqref{eqn:dipic_dilution},
\begin{equation*}
\begin{split}
v_L&=\frac{v_{soln} \times \text{[L]}_0}{\text{[L]}_{stock}}\\
&=\frac{500\text{ {\textmu}L} \times 0.10\text{ M}}{0.4\text{ M}}\\
&=125\text{ {\textmu}L}
\end{split}
\end{equation*}
Equation \eqref{eqn:buff_volume} gives the amount of phosphate buffer, $v_{buff}$, added to the carbonic anhydrase solution (volume $v_{CA}$) before placing in the water bath to equilibrate:
\begin{equation*}
\begin{split}
v_{buff}&=v_{soln}-v_{CA}-v_L\\
&=500\text{ {\textmu}L}-250\text{ {\textmu}L}-125\text{ {\textmu}L}\\
&=125\text{ {\textmu}L}
\end{split}
\end{equation*}

As Equation \eqref{eqn:baseline_correction} demonstrates, a baseline correction is needed to convert the raw absorbance velocity, $\left(\frac{dA}{dt}\right)_{t}$, to a proportion which can take the place of $F_\text{CA$\cdot$Zn}$ in Equation \eqref{eqn:integrated_rate_law_frac}. $\left(\frac{dA}{dt}\right)_{uncat}$ was found to be $3.18\times10^{-5}\ sec^{-1}$. One data point that was measured for the 0.10 M dipic concentration experiment was $\left(\frac{dA}{dt}\right)_{t=1.08\ min}=0.00295\ sec^{-1}$. Therefore, (\textbf{note: consider changing t in prior equation to be less ambiguous})
\begin{equation*}
\begin{split}
ln \left(\frac{dA}{dt}\right)_{cat}
&= ln \left( \frac{ \left (\frac{dA}{dt}\right)_{t} }{ \left (\frac{dA}{dt}\right)_{uncat} } \right) \\
&= ln \left( \frac{ 0.00295\ sec^{-1} }{ 3.18\times10^{-5}\ sec^{-1} } \right) \\
&= 4.530
\end{split}
\end{equation*}

Recall from Equation \eqref{eqn:integrated_rate_law_frac} that $-k_{obs}$ is the slope obtained from a least-squares regression of multiple assay $ln \left(\frac{dA}{dt}\right)_{cat}$ readings versus CA/dipic reaction time. For 0.10 M dipic, this was found to be $-k_{obs}=-0.090 min^{-1}$ (so $k_{obs}=0.090 min^{-1})$.

Equation \eqref{eqn:kobs_slope_reciprocal} implies that a least-squares regression of $\frac{1}{k_{obs}}$ versus $\frac{1}{\text{[L]}_0}$ gives a slope, $m=\frac{1}{k_{d}K_{EML}}$, and intercept, $b=\frac{1}{k_{d}}$, which can be used to calculate $k_d$ and $K_{EML}$ for the dipic/CA reaction. Empirically, $m$ was found to have a value of $0.5372\ M\cdot{mins}$ ($0.50\ M\cdot{mins}$ rounded to accepted precision from error analysis), and b was found to be $8.513\ mins$ ($9.0\ mins$ rounded to accepted precision from error analysis). Thus,
\begin{equation}\label{eqn:samp_calc_keml}
\begin{split}
K_{EML}&=\left(\frac{b}{m}\right){\ }M^{-1}\\
&=\frac{\left (9.0{\ }mins\right )}{\left (0.5{\ }M\cdot{mins}\right )}\\
&=20\ M^{-1}\\
\end{split}
\end{equation}
\begin{equation}\label{eqn:samp_calc_kd}
\begin{split}
k_{d}&=\left(\frac{1}{b}\right){\ }mins^{-1}\\
&=\frac{1}{\left (9.0{\ }mins\right )}\\
&=0.1\ mins^{-1}\\
\end{split}
\end{equation}