\section{Error Analysis}
To determine $k_{obs}$, readings of $ln \left(\frac{dA}{dt}\right)_{cat}$ were plotted against CA/dipic reaction time, and a least squares regression was performed. From the analysis of 0.10 M dipic, $m=-0.0936{\ }{mins}^{-1}$ and $s_m=0.0207{\ }{mins}^{-1}$. There were 5 data points taken while active CA enzyme activity was detected, so 3 degrees of freedom were used for the 95\% confidence t-factor, $t_{95}=3.182$. From this, a 95\% confidence interval was calculated:
\begin{equation*}
\begin{split}
-k_{obs}&=\left (-0.0936\pm\frac{0.0207\times{3.182}}{\sqrt{5}}\right ){\ }{mins}^{-1}\\
&=\left (-0.090\pm0.029\right ){\ }{mins}^{-1}
\end{split}
\end{equation*}

To calculate $k_{d}$ and $K_{EML}$, values of $\frac{1}{k_{obs}}$ retrieved from all five experiments were plotted against $\frac{1}{\text{[L]}_0}$. A least squares regression determined $m=0.5372{\ }mins\cdot{M}$, $s_m=0.1675{\ }mins\cdot{M}$, $b=8.513{\ }{mins}$, and $s_b=5.507{\ }{mins}$. Since 5 runs were performed, there were 3 degrees of freedom. The 95\% confidence t-factor was $t_{95}=3.182$. Therefore,
\begin{equation}\label{eqn:err_analysis_m}
\begin{split}
m_{95}&=\left (0.5372\pm\frac{0.1675\times{3.182}}{\sqrt{5}}\right ){\ }mins\cdot{M}\\
&=\left (0.50\pm0.24\right ){\ }mins\cdot{M}
\end{split}
\end{equation}
\begin{equation}\label{eqn:err_analysis_b}
\begin{split}
b_{95}&=\left (8.513\pm\frac{5.507\times{3.182}}{\sqrt{5}}\right ){\ }{mins}\\
&=\left (9\pm8\right ){\ }{mins}
 \end{split}
\end{equation}

Equation \eqref{eqn:calculating_keml} shows how to calculate $K_{EML}$ from the slope and intercept determined by Equations \eqref{eqn:err_analysis_m} and \eqref{eqn:err_analysis_b}:
\begin{equation*}
K_{EML}=\left (\frac{b}{m}\pm\lambda_{K_{EML}}\right ){\ }M^{-1}
\end{equation*}
Therefore, a 95\% confidence interval can be established as follows:
\begin{equation}\label{eqn:err_anal_keml}
\begin{split}
\lambda_{K_{EML}}&=\sqrt{\left (\frac{\partial{\ }K_{EML}}{\partial{\ }m}\right )^{2}\lambda_{m}^{2}+\left (\frac{\partial{\ }K_{EML}}{\partial{\ }b}\right )^{2}\lambda_{b}^{2}} \\
&=\sqrt{\left (-{\ }\frac{b}{m^{2}}\right )^{2}\lambda_{m}^{2}+\left (\frac{1}{m}\right )^{2}\lambda_{b}^{2}} \\
&=\sqrt{\left (-{\ }\frac{\left (9.0{\ }mins\right )}{\left (0.5{\ }M\cdot{mins}\right )^{2}}\right )^2\times\left (0.24{\ }M\cdot{mins}\right )^2+\left (\frac{1}{\left (0.5{\ }M\cdot{mins}\right )}\right )^2\times\left (8.0{\ }mins\right )^2} \\
&=18\ M^{-1}
\end{split}
\end{equation}

To calculate $k_d$, use the y-intercept of the line passing through a plot of $\frac{1}{k_{obs}}$ versus $\frac{1}{\text{[L]}_0}$ as determined by Equation \eqref{eqn:err_analysis_b}. From Equation \eqref{eqn:calculating_kd},
\begin{equation*}
k_{d}=\left (\frac{1}{b}\pm\lambda_{k_{d}}\right ){\ }mins^{-1}
\end{equation*}
These can be combined to give a 95\% confidence interval:
\begin{equation}\label{eqn:err_anal_kd}
\begin{split}
\lambda_{k_{d}}&=\sqrt{\left (\frac{\partial{\ }k_{d}}{\partial{\ }b}\right )^{2}\lambda_{b}^{2}}\\
&=\sqrt{\left (-{\ }\frac{1}{b^{2}}\right )^{2}\lambda_{b}^{2}}\\
&=\sqrt{\left (-{\ }\frac{1}{\left (9.0{\ }\right )^{2}}\right )^2\times\left (8.0{\ }\right )^2}\\
&=0.10\ mins^{-1}
\end{split}
\end{equation}
